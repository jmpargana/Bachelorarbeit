\chapter{Introduction}
\label{cha:introduction}


This project was conducted following five steps. Firstly, by analysing the historical background related to pedagogy with technologies and researching the related work. The second step was to establish a model with the ideal tool that could satisfy the goal set forth for the project. Third, define an evaluation method to justly judge and assess the efficiency of the implementation. Fourth, select a wide range of individuals that represent various classes of target users and perform the experiment and survey. And finally, draw conclusions and inferences from the resulting data to either prove the point accentuated in the Abstract section or to reevaluate and share the yielded knowledge with the academic world.

\section{Background}
As mentioned in the Abstract section, this projects sets forth with the important judgement that learning practices follow a seemingly delayed convention as to what tools and systems should be used. Modern technology adoption follows several phases or life cycles which obey the Roger's bell. As shown statistically \cite{areviewoftamstudiesinthefieldofeducation}, academic adoption of technology generally materializes in the closing end of the wave. This phenomenon has many factors and reasons that justify the norm. For instance, if we use teachers as an example, who mostly are accustomed to follow a model, of which they have come to feel experienced with, the change to different practices and principles has a high probability to jeopardize the productivity of the teaching process. Of course the point of this dissertation is not to question the patterns of teachers and the reasons why many choose to maintain the customs they feel most confident with, but instead to simply point out, that the learning process is an exercise carried out by and for the student, who has the choice and flexibility to undertake any system of learning and should as such be exposed to a wide variety. It is my opinion that students have a predisposition to discard newer technologies for their teachers disbelief's, which might end up having the unfortunate consequence of depriving a student from a more efficient practicing system than any other they may be familiar with.

The development of a Smart Speaker as Studying Assistant sets out to defy this norm and prove that early technology adoption might boost the learning productivity, if the software is easy enough to use. The Studying Assistant skill developed for Alexa devices follows no bleeding edge theories on efficient pedagogy, but merely simplifies the process to repeat and replicate a practicing dialog of asking and answering simple multiple choice questions, specially in parallel with any other activities that might not disrupt the oral communication with the device.

\section{Related Work}
This thesis exposes and uncovers three important learning fields that represent a vastly unknown part
of the psychology behind teaching and learning processes and that can be better explored using 
technologies that try to connect learners socially
and tackle more unorthodox learning processes that already take place passively.
The resulting product will be assessed and the efficiency of it will answer the research questions
set out in this chapter.



\section{Research Questions}
The research questions related with this thesis work are: 

\begin{enumerate}
	
\item What is the learner's attitude towards less traditional learning tools, that:
    \begin{enumerate}
        \item take advantage of more informal practices?
        \item use innovative technologies?
        \item try to combine social, informal and micro learning in one place?
    \end{enumerate}

\item Are users aware of how important/efficient this less traditional ways of learning are?
\item Are users aware of how important/efficient use of technology to help the learning process is?

\end{enumerate}

\section{Outline}
The remaining part of the thesis is organized as follows:

In the second chapter, we discuss the related work, namely the above mentioned fields of study 
that try to be addressed by the application, \nameref{section:cops}, \nameref{section:informal_learning}, \nameref{section:tacit_knowledge} and \nameref{section:micro} as well as the base model to assess the validity of the 
resulting product, the \nameref{section:tam}.
The third chapter explains the implementation of the developed application, and how its architecture
was planned and which technologies were selected.
In the fourth chapter the evaluation process is reviewed from which population was selected and which
methodology used to assess the productivity of the product.
In the fifth chapter we discuss the results assessed in the previous chapter and how that information
could help us answer the research questions and finally in the sixth chapter a conclusion is drawn
to the application's efficiency and the general public attitude towards it.