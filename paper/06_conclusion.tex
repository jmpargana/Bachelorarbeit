\chapter{Outlook}
\label{cha:outlook}

\section{Conclusion}

In conclusion, two goals were achieved with this thesis. One, a eLearning tool set out to assist learners
with a very contemporary form of technology, namely a smart speaker, was created to cover the type of
informal learning that could only undertake using it. And two, an evaluation of the impact of such a 
technological tool amongst users of all kinds, analyzing how adoption and transition to newer learning
procedures can occur.

The application, although not complete in terms of all the potential features it could integrate to 
represent the ideal tool comprising a framework with the capabilities to form communities of practice, 
be used in all types of informal learning, carrying out a smooth transition from tacit knowledge practicing
to a more explicit one and ultimately conveying well packed micro content, it certainly fulfills its
first and main purpose, such as act as a simplifier of the type of question-based learning done
with multiple choice quizzing. This aspect can be proved by reviewing the positive results all the
participants in the experiment delivered.

On the other hand, the evaluation and assessment showed that the previous attitude towards technology
has a very large weight over the whole procedure, and that adoption for those less willingly is at best
gradual and slow. The results of the survey are not conclusive, in the sense that the experiment was
most likely fractional to the numbers and methods needed to assess it more definitely, but the inference
that this aspect plays the largest role in the adoption of new technology, especially for learning purposes
is undeniable.



\section{Future Work}
For the future remain two steps that can be improve the verdict. First, the missing and suggested features, 
both the ones exposed in the \nameref{subsection:future} section of the \nameref{cha:implementation} chapter,
as well as the ones suggested by the users, for example a more interactive social network that would
implement a proper community of practice. And secondly, perform a more vast evaluation that could follow
both more users of each of the defined categories/classes \ref{section:sampling}, as well as through a longer
period of time following more learning episodes. 
